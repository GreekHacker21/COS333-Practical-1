\documentclass[12pt]{article}
\usepackage{amsmath}
\usepackage{graphicx}
\usepackage{caption}
\usepackage{subcaption}
\usepackage{booktabs}
\usepackage{float}
\usepackage[utf8]{inputenc}
\usepackage{geometry}
\usepackage{multirow}
\usepackage{setspace}
\usepackage{parskip}
\usepackage{svg}
\usepackage[bottom]{footmisc}
\usepackage{tikz}
\usepackage[section]{placeins}
\usepackage{chngcntr}
\counterwithin{figure}{section}
\usepackage{indentfirst}
\geometry{a4paper, margin=1in}

%------------------------------------------------------------------
% COVER PAGE DETAILS
%------------------------------------------------------------------
\def \LOGOPATH {assets/tuks.png}
\def \DEPARTEMENT {Department of Computer Science}
\def \COURSENUM {COS333}
\def \COURSENAME {Programming Languages}
\def \REPORTTITLE {Research Questions Report}
\def \STUDENTNAME {Jason Antalis}
\def \STUDENTID {19141859}

%------------------------------------------------------------------
% TITLE
%------------------------------------------------------------------
\title{
    \includegraphics[width=0.7\textwidth]{\LOGOPATH} \\
    \begin{center}
        \hfill \\
        \Large{\DEPARTEMENT} \\
        \Large{\COURSENUM\;-\;\COURSENAME} \\
        \vfill
        \textbf{\LARGE{\REPORTTITLE}}
    \end{center}
    \mbox{}
    \vfill
    \date{}
    \begin{flushleft}
        \Large{\textbf{Prepared by:} \STUDENTNAME} \\
        \Large{\textbf{Student number:} \STUDENTID} \\
        \Large{\textbf{Date:} \today}
    \end{flushleft}
}


%------------------------------------------------------------------
% DOCUMENT START
%------------------------------------------------------------------
\begin{document}
\maketitle

\newpage
\tableofcontents
\newpage

%------------------------------------------------------------------
% Question 1
%------------------------------------------------------------------
\section{Esoteric programming language \hfill \text{Question 1}}
\subsection{Explanation}
An esoteric programming language (or esolang) is a computer programming language that is not designed to offer an efficient or elegant solution to computational problems but to explore the basic ideas behind computation theory or have fun and create a completely unique programming language.
\cite{Question1Explain1}
\cite{Question1Explain2}
%------------------------------------------------------------------
% Question 2
%------------------------------------------------------------------
\section{Categories \hfill \text{Question 2}}
\subsection{Funges}
\subsubsection{Advantages}
\subsubsection{Disadvantages}
\subsection{Stateful encoding lamguages}
\subsubsection{Advantages}
\subsubsection{Disadvantages}

%------------------------------------------------------------------
% Question 3
%------------------------------------------------------------------
\section{Chosen languages \hfill \text{Question 3}}
\subsection{Chef}
\subsubsection{Description}
\subsubsection{Code Snippet}
\subsection{Language 2}
\subsubsection{Description}
\subsubsection{Code Snippet}

%------------------------------------------------------------------
% Question 4
%------------------------------------------------------------------
\section{Design by Contract \hfill \text{Question 4}}
\subsection{Explanation}
\subsection{Language Support}
\begin{itemize}
    \item Effiel
    \cite{Question4Point2}
    \item Racket
    \cite{Question4Point1}
\end{itemize}
\newpage

%------------------------------------------------------------------
% REFERENCES
%------------------------------------------------------------------
\bibliographystyle{plain}
\bibliography{references}
\end{document}
